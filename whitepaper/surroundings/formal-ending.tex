\chapter{Work division}
    As this was a purely research project, it is hard to
    put a clear line defining who made what at concept level.
    We all have learned many things and spent a lot of time
    on discussion of possible solutions and problem,
    so we all contributed to every part of project.
    However, two groups can be separated
    with greater contribution to other parts of the project.\\
    Face recognition -- Tomasz Garbus, Jan Ludziejewski.\\
    Liveness detection -- Dominik Klemba, Łukasz Raszkiewicz.\\
    But we weren't limited to our main subjects.
    More exact informations can be find below, but keep the note above in mind.

    \section{Tomasz Garbus}
        \begin{itemize}
            \item Implementation of convolutional neural network in Tensorflow.
            \item Building a pipeline for training CNN:
            \begin{itemize}
                \item Splitting dataset into train and test set.
                \item Code for loading dataset.
                \item Partial preprocessing (trimming faces, depth normalization,
                      data augmentation), caching preprocessed input data for future experiments.
            \end{itemize}
            \item Conducting experiments with CNN.
        \end{itemize}

    \section{Dominik Klemba}
        \begin{itemize}
            \item TODO
        \end{itemize}

    \section{Jan Ludziejewski}
        \begin{itemize}
            \item TODO
        \end{itemize}

    \section{Łukasz Raszkiewicz}
        \begin{itemize}
            \item Implementation of most code using Kinect.
            \begin{itemize}
                \item Libkinect -- a library which gives a unified interface for
                using both Kinect v1 and Kinect v2 cameras.
                \item Live display program -- shows the video feeds from Kinect,
                allows saving frames to hard drive, limiting FPS.
                \item File display program for depth and IR photos in our file format.
                \item Thumbnailer for depth and IR photos.
            \end{itemize}
            \item Creating the depth photo database.
            \begin{itemize}
                \item Taking photos of volunteers.
                \item Legal consultations.
            \end{itemize}
            \item Skin recognition using Kinect -- implementing the programs required
            for testing the idea and implementing the first attempts.
            \item Skin recognition prototype
            \begin{itemize}
                \item Research and purchase of required IR diodes.
                \item Learning Arduino programming and how to build circuits.
                \item Researching methods to remotely take photos from a phone's camera.
                \item Writing an Arudino program and a Python script to synchronize
                taking photos in different wavelengths.
                \item Building the cardboard setup.
                \item Taking photos of various objects in different wavelengths.
                \item Attempting various methods to recognize skin on those photos.
            \end{itemize}
            \item Creating appropriate parts of the final presentation.
            \item Writing appropriate parts of this paper.
        \end{itemize}

\chapter{CD content}
    The included CD contains the source code of all final algorithms
    and models, as well as a lot of code to run some experiments
    which results are described in the paper.
    However, the code of abandoned or improved ideas can be missing.

    \begin{itemize}
        \item libkinect
        \begin{itemize}
            \item AAA stuff
            \item BBB stuff
        \end{itemize}
        \item skin-recognition
        \item face-auth
        \item whitepaper
    \end{itemize}
