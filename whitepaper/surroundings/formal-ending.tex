\chapter{Work division}
    As this was a purely research project, it is hard to
    put a clear line defining who made what at concept level.
    We all have learned many things and spent a lot of time
    on discussion of possible solutions and problems,
    so we all contributed to every part of project.
    However, two groups can be separated
    with greater contribution to other parts of the project.\\
    Face recognition -- Tomasz Garbus, Jan Ludziejewski.\\
    Liveness detection -- Dominik Klemba, Łukasz Raszkiewicz.\\
    But we were not limited to our main subjects.
    More specific information can be found below, but keep the note above in mind.

    \section{Tomasz Garbus}
        \begin{itemize}
            \item Implementation of convolutional neural network in Tensorflow:
            \begin{itemize}
                \item Two modes: binary and multi-class, customizable size, neurons
                      counts, dropout, etc.
                \item Visualizations:
                    \begin{itemize}
                        \item Learned convolution kernels.
                        \item Classified and misclassified inputs.
                        \item Most activating patches for each convolution kernel.
                    \end{itemize}
                \item Data augmentation: on-the-fly and as preprocessing.
                \item Storing checkpoints.
                \item Returning raw probabilities for ensembling.
            \end{itemize}
            \item Building a pipeline for training CNN:
            \begin{itemize}
                \item Code for loading dataset.
                \item Partial preprocessing (trimming faces, depth normalization),
                      caching preprocessed input data for future experiments.
            \end{itemize}
            \item Obtaining $4$ RGBD face databases for tests before we had our own and
                  converting to our format.
            \item Implementation of face rotation (without angle detection).
            \item Refactoring \texttt{face\_auth} code:
            \begin{itemize}
                \item Big refactor of preprocessing and normalization done with Jan.
                \item Providing and updating templates for experiments.
                \item Providing unified output format for classifiers.
            \end{itemize}
            \item Legal consultation.
            \item Choosing score measures for classifiers.
            \item Running final experiments (and choosing the best performing CNN configuration).
            \item Measuring relation between frames used in authentication and
                  classification quality.
            \item Creating appropriate parts of the final presentation.
            \item Writing appropriate parts of this paper.
        \end{itemize}

    \section{Dominik Klemba}
        \begin{itemize}
            \item Work with C++ code.
            \begin{itemize}
                \item First version of GUI.
                \item First version of processing depth pictures.
                \item Implementation of basic types -- mostly matrix class.
                \item Small works and improvements, as well as testing code.
            \end{itemize}
            \item Research in liveness detection.
            \item Infrared skin recognition.
            \begin{itemize}
                \item Extensive research in skin detection problem.
                \item Tools to display detected pixels.
                \item Models to detect skin in photos from the prototype.
            \end{itemize}
            \item Creating small part of the database.
            \item Face authentication.
            \begin{itemize}
                \item RGB skin recognition heuristic.
                \item Considerations about evaluation of model's quality.
                \item Testing code.
            \end{itemize}
            \item Creating appropriate parts of the final presentation.
            \item Writing appropriate parts of this paper.
        \end{itemize}

    \section{Jan Ludziejewski}
        \begin{itemize}
            \item Research on face detection, checking preprocessing possibilities.
            \item Face angle detection and rotation, testing methods for smoothing and retrieving missing information.
            \item SVM and Extra Trees classifiers for binary and multi-class
            classification with Histogram of Oriented Gradients preprocessing.
            \item Preprocessing for both classifiers and neural network.
            \item Tuning classifiers and preprocessing parameters.
            \item Ensembling of all methods.
            \item Tests of ensemble methods and voting weights.
            \item Restructuring tests and architecture, creating Face class, planning modules and refactoring code.
            \item Tools for face position visualizations, image concatenation etc.
            \item Creating appropriate parts of the final presentation.
	          \item Legal consultations.
            \item Writing appropriate parts of this paper.
        \end{itemize}

    \section{Łukasz Raszkiewicz}
        \begin{itemize}
            \item Implementation of most code using Kinect.
            \begin{itemize}
                \item Libkinect -- a library which gives a unified interface for
                      using both Kinect v1 and Kinect v2 cameras.
                \item Live display program -- shows the video feeds from Kinect,
                      allows saving frames to hard drive, limiting FPS.
                \item File display program for depth and IR photos in our file format.
                \item Thumbnailer for depth and IR photos.
            \end{itemize}
            \item Liveness detection -- pulse detection research.
            \item Creating the depth photo database.
            \begin{itemize}
                \item Taking photos of volunteers.
                \item Legal consultations.
            \end{itemize}
            \item Skin recognition using Kinect -- implementing the programs required
                  for testing the idea and implementing the first attempts.
            \item Skin recognition prototype
            \begin{itemize}
                \item Research and purchase of required IR diodes.
                \item Learning Arduino programming and how to build circuits.
                \item Researching methods to remotely take photos from a phone's camera.
                \item Writing an Arudino program and a Python script to synchronize
                      taking photos in different wavelengths.
                \item Building the cardboard setup.
                \item Taking photos of various objects in different wavelengths.
                \item Attempting various methods to recognize skin on those photos.
            \end{itemize}
            \item Setting up GitLab CI to automatically compile this paper with each commit.
            \item Creating appropriate parts of the final presentation.
            \item Writing appropriate parts of this paper.
        \end{itemize}


\chapter{CD content}
    The included CD contains the source code of all final algorithms
    and models, as well as a lot of code to run some experiments
    which results are described in the paper.
    However, the code of abandoned or improved ideas can be missing.

    \begin{itemize}
        \item \texttt{libkinect/}
        \begin{itemize}
            \item \texttt{libkinect.hpp} -- a library which gives a unified interface
                  for receiving frames from both Kinect v1 and Kinect v2.
            \item \texttt{live\_display} -- a program displaying live feeds from a Kinect
                  camera, also allowing to save received frames.
            \item \texttt{file\_display} -- a program displaying a depth or IR photo saved
                  by \texttt{live\_display}.
            \item \texttt{thumbnailer} -- a thumbnailer that can be used with graphical
                  file managers to show thumbnails for depth and IR photos.
        \end{itemize}
        \item \texttt{face\_auth/}
        \begin{itemize}
            \item \texttt{classifiers/} -- implementations of neural network,
                  HOG-based classifier and ClassificationResults class for
                  unified classification results.
            \item \texttt{common/} -- constants, utilities for image manipulation and plots,
                  DBHelper class for unified database loading.
            \item \texttt{controller/} -- Controller-layer code for face normalization.
            \item \texttt{experiments/} -- A playground. Each experiment is assigned its
                  own independent subdirectory. Also contains directory \texttt{templates}
                  with some shared helpers.
            \item \texttt{face\_rotation/} -- All of code concerning face normalization, rotation
                  and angle detection.
            \item \texttt{model/} -- Model-layer code -- Face class.
        \end{itemize}
        \item \texttt{skin\_recognition/}
        \begin{itemize}
            \item \texttt{prototype/take\_ir\_photos.py} -- a program synchronizing
                  lighting up the diodes and taking photos on the phone.
            \item \texttt{prototype/arduino\_project/} -- an Arudino project
                  compatible with the photo taking script.
            \item \texttt{find\_model.py} -- calculating a skin detection model from
                  photos taken with 850, 890, and 940nm IR light.
            \item \texttt{rgb\_model.py} -- RGB skin detection heuristic.
        \end{itemize}
        \item \texttt{whitepaper/} -- sources for this paper.
        \begin{itemize}
            \item \texttt{mim\_paper.tex} -- main LaTeX file.
            \item \texttt{mim\_paper.pdf} -- a compiled version.
        \end{itemize}
        \item \texttt{presentation/} -- slides and videos that we showed on our
              final presentation.
        \item \texttt{poster.pdf} -- our poster.
    \end{itemize}
