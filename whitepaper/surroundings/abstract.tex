\begin{abstract}
    Currently, commonly used authentication methods are becoming obsolete,
    due to the lack of convenience, adversarial technology improvements
    or easy to make user errors.
    We believe that biometric authentication has the potential to become
    one of the safest, most convenient and most efficient methods.
    In this paper we describe our efforts to improve face recognition
    using a camera with depth-perception and infrared capabilities, as well as
    our search for new liveness detection methods such as skin detection
    using multispectral imaging.
    This text is mostly written with mobile solutions in mind,
    but the results could also be applied in other settings.
\end{abstract}

% Powszechnie używane metody uwierzytelniania stają się przestarzałe. Jest to spowodowane brakiem wygody użytkowania, coraz wydajniejszymi metodami łamania, jak również podatnością na błędy użytkowników. Wierzymy, że zabezpieczenia biometryczne mają potencjał, by stać się jednymi z najbezpieczniejszych, najwygodniejszych oraz najbardziej wydajnych. W tej pracy opisaliśmy nasze próby ulepszenia rozpoznawania twarzy przy użyciu kamery głębi oraz podczerwieni, a także poszukiwania metod wykrywania życia - między innymi opis próby wykrywania skóry przy użyciu zdjęć multispektralnych. Ta praca została napisana z myślą o urządzeniach mobilnych, jednak jej wyniki mogą znaleźć też inne zastosowania.
