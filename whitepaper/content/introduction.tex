\section{Introduction}
    In the world of authentication, there is always a trade-off between
    convenience and security. We would like to compare commonly used authentication
    methods with regard to both factors and argue that face recognition is more than
    a good balance between the two -- it does very well in terms of both security
    and convenience.

    \subsection*{Conventional authorization methods}
        \subsubsection*{Passwords}
            Passwords come in various forms and cannot be easily placed on a
            security-convenience curve. Their strength is exponentially dependant
            on their length.\\
            The strength and form of a password often depend on the context of its
            use.\\
            For instance, passwords used for unlocking the screen in a mobile
            device tend to be simple and quick to input -- a pattern to draw, a
            $4$ to $8$-digit pin, in rare cases -- an alphanumeric sequence.\\
            Passwords used for bank accounts, on the other hand, can be undeniably
            strong, to the point of being a randomly generated sequence of lower
            and upper letters, digits and special characters.\\
            With those two extreme examples in mind, we can conclude that passwords
            can be very secure and they can be very convenient. But that does not
            imply that passwords can be very secure and very convenient at the
            same time.\\
            Another thing to keep in mind is that passwords are an authentication
            method that rely on user's \textit{knowledge} -- knowledge of the
            secret code and the ability to remember it at all times. That, however,
            is slowly changing. Password managers and credential managers are taking
            the responsibility of not only storing the passwords -- but generating
            them too.

        \subsubsection*{Pendrive keys, tokens and magnetic cards}
            Sometimes to protect valuable resources,
            something stronger than standard password is needed.
            One possibility is pendrive with secret keys.
            It's obviously very secure method...
            as long as that very pendrive is secure.
            By stealing it, thief gains access to resources.
            Someone clever may even copy pendrive, and
            owner will not notice problem,
            until it will be too late.

            Same problems are with tokens.
            It's true that often felon cannot copy that,
            but this is ''compensated'' by length of token's passwords.
            Few digits long? Not realy strong.
            There are many ideas to overcome that problem,
            but stealing token is still an option.

            Magnetic cards are simillar, so we only want to notice,
            that it's possible to receive access if there are
            two attackers and one is near owner.
            Attacer doesn't have to reach card, it's enough
            for him to stand near by. %TODO taki na karty kredytowe itp. linki

    \subsection{Bio authentication}
        Above we see problems with one simple source.
        We have a group of people who should have access to resources,
        but instead of writing list of them, we
        say ''everyone who has/knows...'' has access.
        It's simply turning one specific problem into another,
        more general. Therfore unauthorized felons may
        access resource.\\
        Bio authentication is answer to inconvenience of that change.
        Why we should redefine main problem? Maybe we
        can just list authorized persons?
        Actually it's not simple, therefore we should
        look for secure possibilities to avoid redefinition.\\
        As stealing someone's face will always be hard,
        bio-authentication may become not only the most
        convenient, but also one of the safest
        authentication methods.\\
        Is it now? Let's look closer.

        \subsubsection*{Fingerprint}
            Mankind use fingerprints to identify specific person
            for long tine. %TODO source needed
            However, idea of faking that is also old. %TODO [PL] Sherlock Holmes mial opowiadanie z
            % fake odciskiem palca w krwii (stempel byl zrobiony z wosku, ktorego dotknal wrabiany).
            % Warto zalinkowac.
            And now it's easy. %TODO Sourceneeded.
            Hackers gain access to phones
            using fake fingerprints. %TODO Source.
            There is also no good way
            to deal with it.
            *[Some explanation why?]* %TODO

        \subsubsection{Face recognition}
            Everyone has own, unique face, as he has unique fingerprint.
            So here idea is same,
            but faking face may be little bit harder.
            However, pictures and masks may be very decent solutions
            for hakers. %TODO source
            Therfore we shouldn't stop here.

        \subsubsection{Face authentication -- liveness detection}
            Line between face recognition and
            good face authentication is very thin.
            If we assume that we have perfect face recognition,
            we still need something... (Because, you know, on your picture is your face...)
            like liveness detection --
            system checking if there is real human, not picture, mask or mannequin.
            With both perfect systems, only way to access resurce will be
            conviction someone with acess to help us -- but this last problem
            we leave for someone else to solve.

    \subsection{Results}
        Brief explanation of work results.
        What someone may find in this paper,
        and what look for in other materials.
