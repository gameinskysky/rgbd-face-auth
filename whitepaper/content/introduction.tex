\section{Introduction}
    Wide explanation of [security improvement] motivations.

    % TODO Add links to examples of attacks.
    \subsection*{Old style autorization}
        World knows many authentication methods.
        Let's look at some known from long time
        to understand their proses and cons.
        To see why they are getting old.

        \subsubsection*{Passwords}
            It's easy to improve standard password's strength,
            cause longer (random) password is better password
            and quality increase is exponential.
            Hash function may create strength limit,
            but this problem is very easy to overcome on development level,
            therefore normally possible strength is better tan enough.
            So, it may be quite good authentication method.\\
            However, people by themselves simplify passwords to
            (sometimes literally) the level of picking the right picture,
            making strong system weak.\\
            Looking closer on systems avaliable in modern phones,
            we are able to understand how user's need of comfort
            had impact on security.
            If you have to mark three out of $k$ pictures, there
            is only $\binom{k}{3} = \frac{k(k-1)(k-2)}{6} = O(k^3)$ options,
            so for $k=9$ there is only 84 options.
            Even with big overhead to one try, it's esy to break.
            Same with popular (at phones) patterns out of
            nine dots. Looking at simple implementations
            limit of possibilities is near $\sum_{k=2}^9k! = 409\,112$,
            % TODO linki??
            and it's not big number for computers.\\
            Not only brute-force may be problem here.
            It's easy for a bystander to see the password
            entered in such a manner. Same
            problem with standard password.
            Especially that we live in a world full of cameras.

        \subsubsection*{Pendrive keys, tokens and magnetic cards}
            Sometimes to protect valuable resources,
            something stronger than standard password is needed.
            One possibility is pendrive with secret keys.
            It's obviously very secure method...
            as long as that very pendrive is secure.
            By stealing it, thief gains access to resources.
            Someone clever may even copy pendrive, and
            owner will not notice problem,
            until it will be too late.

            Same problems are with tokens.
            It's true that often felon cannot copy that,
            but this is ''compensated'' by length of token's passwords.
            Few digits long? Not realy strong.
            There are many ideas to overcome that problem,
            but stealing token is still an option.

            Magnetic cards are simillar, so we only want to notice,
            that it's possible to receive access if there are
            two attackers and one is near owner.
            Attacer doesn't have to reach card, it's enough
            for him to stand near by. %TODO taki na karty kredytowe itp. linki

    \subsection{Bio authentication}
        Above we see problems with one simple source.
        We have a group of people who should have access to resources,
        but instead of writing list of them, we
        say ''everyone who has/knows...'' has access.
        It's simply turning one specific problem into another,
        more general. Therfore unauthorized felons may
        access resource.\\
        Bio authentication is answer to inconvenience of that change.
        Why we should redefine main problem? Maybe we
        can just list authorized persons?
        Actually it's not simple, therefore we should
        look for secure possibilities to avoid redefinition.\\
        As stealing someone's face will always be hard,
        bio-authentication may become not only the most
        convenient, but also one of the safest
        authentication methods.\\
        Is it now? Let's look closer.

        \subsubsection*{Fingerprint}
            Mankind use fingerprints to identify specific person
            for long tine. %TODO source needed
            However, idea of faking that is also old. %TODO [PL] Sherlock Holmes mial opowiadanie z
            % fake odciskiem palca w krwii (stempel byl zrobiony z wosku, ktorego dotknal wrabiany).
            % Warto zalinkowac.
            And now it's easy. %TODO Sourceneeded.
            Hackers gain access to phones
            using fake fingerprints. %TODO Source.
            There is also no good way
            to deal with it.
            *[Some explanation why?]* %TODO

        \subsubsection{Face recognition}
            Everyone has own, unique face, as he has unique fingerprint.
            So here idea is same,
            but faking face may be little bit harder.
            However, pictures and masks may be very decent solutions
            for hakers. %TODO source
            Therfore we shouldn't stop here.

        \subsubsection{Face authentication -- liveness detection}
            Line between face recognition and
            good face authentication is very thin.
            If we assume that we have perfect face recognition,
            we still need something... (Because, you know, on your picture is your face...)
            like liveness detection --
            system checking if there is real human, not picture, mask or mannequin.
            With both perfect systems, only way to access resurce will be
            conviction someone with acess to help us -- but this last problem
            we leave for someone else to solve.

    \subsection{Results}
        Brief explanation of work results.
        What someone may find in this paper,
        and what look for in other materials.
