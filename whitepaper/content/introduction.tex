\section{Introduction}
    In the world of authentication, there is always a trade-off between
    convenience and security. We would like to compare commonly used authentication
    methods with regard to both factors and argue that face recognition is more than
    a good balance between the two -- it does very well in terms of both security
    and convenience.

    \subsection{Conventional authorization methods}
        \subsubsection*{Passwords}
            Passwords come in various forms and cannot be easily placed on a
            security-convenience curve. Their strength is exponentially dependant
            on their length.\\
            The strength and form of a password often depend on the context of its
            use.\\
            For instance, passwords used for unlocking the screen in a mobile
            device tend to be simple and quick to input -- a pattern to draw, a
            $4$ to $8$-digit pin, in rare cases -- an alphanumeric sequence.\\
            Passwords used for bank accounts, on the other hand, can be undeniably
            strong, to the point of being a randomly generated sequence of lower
            and upper letters, digits and special characters.\\
            With those two extreme examples in mind, we can conclude that passwords
            can be very secure and they can be very convenient. But that does not
            imply that passwords can be very secure and very convenient at the
            same time.\\
            Another thing to keep in mind is that passwords are an authentication
            method that relies on user's \textit{knowledge} -- knowledge of the
            secret code and the ability to remember it at all times.\\
            That, however, is changing. Password managers and credential managers have been taking
            the responsibility of not only storing the passwords but generating
            them too. The user does not have to know their password -- it is sufficient
            that the password manager knows it and that the user is in \textit{control}
            of password manager. It is a good defense against phishing attacks,
            assuming the user can be tricked into giving their password more
            easily than the password manager. \\


        \subsubsection*{Hardware tokens and magnetic cards}
            Sometimes to protect valuable resources,
            something stronger than standard password is needed.
            Hardware tokens such as U2F-like tokens or password-protected RSA SecurID
            and magnetic cards are good choices then.
            Those authorization methods rely on user's
            \textit{possession} of said token or magnetic card. To
            gain access to the victims resources, an attacker must take gain
            possession of the physical key, whether it is a U2F- or SecurID-like token.
            Magnetic cards are more vulnerable to attacks -- it is possible to
            conduct an attack without obtaining the card itself, but instead
            being within its close range.

    \subsection{Biometric authentication}
        Conventional authorization methods are based on \textit{possession},
        \textit{knowledge} or \textit{control} over some resource. Usually, the
        resources in question are copyable.
        We have no intention of undermining such methods, because they can be
        extremely secure if designed well.

        Instead, we would like to focus on another paradigm of authentication,
        which relies on \textit{being} -- being the right person to authorize.
        With this approach, the authorization keys -- users -- are unique and non-copyable. However,
        their uniqueness is as good as the authorization system's ability to
        distinguish between them.

        Biometric authentication is a set of techniques striving to recognize a
        human being by their appearance, body or behavior.

        \subsubsection*{Fingerprints}
            % If we want some link to literature, I'd go with Janusz Zajdel's Limes
            % Inferior. It has the exact theme of chopping off fingers as a form
            % of mugging - in the created world each fingerprint had some
            % "foodstamps" assigned. What's interesting about this example,
            % the fingerprints are checked electronically in the book and it
            % was published in 1982. -tomek
            Fingerprints have been used for a long time for identifying their
            owner, especially in criminology. They are perfectly fine in terms of
            distinguishing people as everyone's fingerprint is unique. Moreover,
            even if there happened to be two people with identical fingerprints,
            it would be an extremely unlikely coincidence for one of them to try
            to hack the other.\\
            Unfortunately, while users are a non-copyable resource, their fingerprints
            are indeed copyable.

        \subsubsection*{Behavioral biometrics}
            Behavioral biometrics have been making their way into
            the banking industry. They are a set of techniques supporting the
            conventional authentication, by learning, and verifying, user's
            preferences and habits, such as (by \citeauthor{behavioral}):
            \begin{itemize}
                \item Behavioral qualities of the input data, such as typing
                      speed or touchscreen interactions and their timing;
                \item Supporting contextual factors of the current transaction,
                      such as user device type, IP address, geolocation; and
                \item The user’s historical behavioral factors, such as the
                      typical timing of user access, prior purchasing or access
                      patterns, etc.
            \end{itemize}
            The downside is, behavioral biometrics are not a standalone solution.
            They are more of an anomaly detection system reinforcing the authentication
            system.

        \subsubsection*{Face recognition}
            Face recognition is an authentication method that is both convenient
            and secure -- in theory. It has been used in smartphones, tablets
            and laptops for several years now, mostly to unlock the device, but
            never really trusted enough for securing more vulnerable resources.\\
            Faces, like fingerprints, are unique, but their uniqueness cannot be
            taken for granted. For instance, a static, two-dimensional picture of
            a face is perfectly copyable.

    \subsection{Liveness detection}
        To design a secure face recognition based authentication system,
        one should ensure the user's face is non-copyable. Only then will the
        system fully comply to our paradigm of relying on \textit{being} instead
        of \textit{possession}.\\
        This can be achieved by adding liveness detection to the authentication
        process. Liveness detection itself can be based on multiple techniques.

        \subsubsection*{Existing methods}
            Various mobile phone manufacturers have implemented methods attempting to
            detect whether the visible face is actually a human and not a photo.
            For example, users were asked to blink or smile to unlock their phones.\\
            Such methods definitely increase the security of face recognition,
            but they're inconvenient to the user -- ideally, one wouldn't need to take
            any action to unlock their device -- and were often easily hacked by having
            additional photos of the owner, even fake ones with the eyes being covered.

        \subsubsection*{Depth channel}
            Using a depth channel in addition to a flat image is a form of liveness
            detection itself.
            When implemented well, such system should block all attempts to authenticate
            using a photo of an authorized person.
            However, it could still be broken using a detailed mask of that face.
            While that would require a lot of effort, it is a realistic way to break
            into for example a stolen phone of an important person.

        \subsubsection*{Pulse detection}
            Determining a person's pulse from a video is not a new idea.
            We have found existing implementations of programs that detected the user's
            pulse, using for example a webcam \cite{pulsedetector}.\\
            Unfortunately, we found out that such liveness detection method would be
            too slow for common authentication purposes.
            Detecting the pulse takes several seconds and requires the face to remain
            in one particular spot through the entire procedure.
            Therefore we we abandoned further research on this subject.

        \subsubsection*{Skin recognition}
            The liveness detection idea which we decided to research is skin recognition.
            With a reliable method to detect whether real skin is visible in the camera,
            it should be nearly impossible to authenticate without actually showing
            the owner's face to the camera.\\
            Our proposed skin recognition ideas will be described later in this paper.

    \subsection{Results}
        In our work, we have focused on skin recognition and face authentication
        with depth channel.

        \subsubsection*{Skin recognition}
            We have developed several skin recognition techniques working with different
            sets of light wavelengths. We have also constructed a simple hardware prototype
            and experimented with machine learning approaches to per-pixel skin recognition.

        \subsubsection*{Face authentication}
            We have constructed a depth + infrared face dataset and evaluated different
            machine learning approaches to face recognition. We have measured the
            effect of authentication time limit on its effectiveness and the trade-off
            between security and convenience.
